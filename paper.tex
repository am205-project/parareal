%        File: paper.tex
%     Created: Sat Dec 06 04:00 PM 2014 E
%
\documentclass[letterpaper,twocolumn,11pt]{article}
% Change the header if you're going to change this.
 
% Possible packages - uncomment to use
\usepackage{amsmath}       % Needed for math stuff
%\usepackage{lastpage}      % Finds last page
\usepackage{amssymb}       % Needed for some math symbols
\usepackage{graphicx}      % Needed for graphics
\usepackage[usenames,dvipsnames]{xcolor} % Needed for graphics and color
%\usepackage{setspace}      % Needed to set single/double spacing
\usepackage{float}         % Better placement of figures & tables
\usepackage{hyperref}           % Can have actual links
\usepackage{mathpazo}
\usepackage[linesnumbered,vlined,ruled]{algorithm2e} % algorithm

%Suggested by TeXworks
\usepackage[utf8]{inputenc} % set input encoding (not needed with XeLaTeX)
\usepackage{verbatim}      % lets you do verbatim text
 
% Sets the page margins to 1 inch each side
\usepackage[margin=1in]{geometry}
\geometry{letterpaper}
%\addtolength{\oddsidemargin}{-0.875in} 
%\addtolength{\evensidemargin}{-0.875in} 
%\addtolength{\textwidth}{1.75in} 
%\addtolength{\topmargin}{-.875in} 
%\addtolength{\textheight}{1.75in}

\frenchspacing

% Uncomment this section if you wish to have a header.
\usepackage{fancyhdr} 
\pagestyle{fancy} 
\renewcommand{\headrulewidth}{0.5pt} % customize the layout... 
\lhead{Chen, W., Sim, B., Shi, A.} \chead{Applied Math 205 Final Project} 
\rhead{Fall 2014} 
\lfoot{} \cfoot{\thepage} \rfoot{}


\begin{document}
\title{Explorations to Optimize the Parareal Algorithm \\to Solve ODEs in Given Applications}
\author{Wesley Chen, Brandon Sim, Andy Shi \\
Harvard University, Applied Math 205 Final Project}
\date{December 15, 2014}

% No indentation for new paragraphs
\setlength\parindent{0pt}

% Space between each new paragraph
\setlength\parskip{2ex}

% Put your stuff here
\twocolumn[
    \begin{@twocolumnfalse}
    \maketitle
    \begin{abstract}
    We implement the parareal algorithm to solve various ODEs and compare in
    terms of accuracy, speedup and efficiency. We explore the effect of using
    differnt solution operators (a coarse and a fine as required by the method)
    The majority of our code is written in python using mpi4py to parallelize.
    To compare to more performance-based languages, we compare certain codes to
    an implementation in C++.
    \end{abstract}
  \end{@twocolumnfalse}
]
\section{Background}

\subsection{Numerical Methods and Parallelization}
Solving systems of differential equations can be a computationally expensive
task. The error of most algorithms scales on the stepsize of the discretization
of time. However, stepsize in time is also proportional to the computation
required. The parareal algorithm, developed by YADA YADA. An alternative would
be to parallelize by space - 

\subsection{Stability of the Parareal}

\subsection{Error of the Parareal}

\section{Methodology}

The general parareal method, as described by YADA YADA can be SEEN BELOW YAR

\begin{algorithm}[t]
    \KwIn{Temporal discretization $t^n = t_0 + n \Delta t, \, n = 1,2,\ldots,N$}
    \KwIn{Coarse scheme $g_{\Delta t}$}
    \KwIn{Finer scheme $g_{\textnormal{fine}}$}
    Compute $u_1^{n+1} = g_{\Delta t}(t^n, u^n)$\;
    Compute the corrections $\delta g^n(u_1^n) = g_{\textnormal{fine}}(t^n,
    u^n) - g_{\Delta t}(t^n, u^n)$ in parallel\;
    Add the prediction and correction terms as $u_2^{n+1} = g_{\Delta
    t}(t^n, u_2^n) + \delta g^n(u_1^n)$\;
    Repeat steps 2 and 3, incrementing the iteration label and using $u_{k+1}^0
    = u^0$ as the initial condition\
    }
 \caption{Parareal}
 \label{alg:parareal}
\end{algorithm}

\subsection{Explored Tests}
We wished to explore certain tests

\section{Results}
We compared various runs of 

\subsection{Comparison to Serial with $g_{fine}$ as \\a Forward Euler with
Smaller $\Delta t$}

\subsection{Comparison to Serial with $g_{fine}$ as \\Higher Order Methods}

\subsection{Comparison to a Forward Euler \\Parallelized by Space}

\subsection{Comparison to C++ Implementations}
	The numpy environment combined with the required structure of MPI required certain type/data structure conversions which will take time.not not cunot cu

\subsection{Summary of Performance}

\section{Discussion}

\section{Conclusion and Future Work}

\section{References}

\end{document}
